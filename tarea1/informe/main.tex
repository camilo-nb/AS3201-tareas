\documentclass[letter, 11pt, twocolumn]{article}

\usepackage[utf8]{inputenc}
\usepackage[spanish]{babel}

\usepackage[left=0.75in,right=0.75in,top=0.75in,bottom=0.75in,columnsep=0.5in]{geometry}
\usepackage{graphicx}
\usepackage{float}
\usepackage{hyperref}

\usepackage{amsmath}
\numberwithin{equation}{section}

\usepackage{physics}

\usepackage{siunitx}
\DeclareSIUnit{\belmilliwatt}{Bm}
\DeclareSIUnit{\dBm}{\deci\belmilliwatt}
\DeclareSIUnit\lightyear{ly}
\usepackage{booktabs}
\usepackage{scrextend}
\def\tablename{Tabla}
\usepackage{stfloats}

\input{listings.tex}

\title{Informe Instrumentación Astronómica}
\author{Camilo Núñez Barra}


\begin{document}
\begin{figure}[H]
	\includegraphics[width=2in]{fcfm_das}
\end{figure}
\begin{center}
	\textbf{\huge{INFORME\\INSTRUMENTACIÓN\\ASTRONÓMICA}}\\\,\\Tarea 1\\\,
\end{center}
\begin{tabular}{rl}
	Estudiante:&Camilo Núñez Barra\\
	RUT:&20.533.326-6\\
	Profesor:&Leonardo Bronfman\\
	Auxiliar:&Paulina Palma\\
	Curso:&AS3201 Astronomía Experimental\\
	Fecha:&\today
\end{tabular}

\section{Introducción}

La visión que se tiene desde la Tierra de la Vía Láctea tiene franjas de oscuridad, que en un principio no se sabía si eran agujeros o materia interviniente. Esto se denomina el dilema de Bernard (1900) y fue resuelto en 1937 por Eddington con la conclusión de que es materia la que está absorbiendo.

La Galaxia está formada principalmente por hidrógeno molecular gaseoso $\textnormal{H}_2$, además de otros gases, polvo, muchas estrellas, sistemas solares, nebulosas y asteroides, todos agrupados gracias a la fuerza de gravedad existente. A menor escala, las nubes frías de hidrógeno gaseoso son importantes porque pueden indicar el lugar de nacimiento de estrellas debido a colapsos gravitacionales, además, en los brazos espirales se forman estrellas de alta masa. A gran escala, estas nubes delinean la estructura de la galaxia.

La molécula de hidrógeno no se puede observar directamente porque la atmósfera bloquea sus transiciones en el ultravioleta, además, es una molécula simétrica y su centro de masa coincide con su centro de carga, por lo que tiene momento dipolar nulo y la probabilidad de transiciones cuadripolares es muy baja.

El monóxido de carbono CO es la segunda molécula más abundante. En cambio, el CO tiene un momento de dipolo fuerte y transiciones  distintas frecuencias. La abundancia del CO relativa al H$_2$ es aproximadamente $10^{-4}$. La densidad de columna de hidrógeno $N(\textnormal{H}_2)$ es proporcional a la temperatura integrada de monóxido de carbono $W(\textnormal{CO})$ mediante el factor de conversión $\chi$ que se puede calcular y vale aproximadamente \SI{2e20}{\per\centi\metre\squared\per\kelvin\per\kilo\metre\second}, por lo tanto se puede usar el CO como trazador de H$_2$ pues mediante la medición de $W(CO)$ se obtiene la densidad de columna de hidrógeno y, junto con el área proyectada de la nube de gas, la cantidad de partículas de hidrógeno, que permite calcular directamente la masa de la nube.

El plano galáctico es donde se concentra la mayor parte de las estrellas de la Galaxia y además pasa por el centro de masa de esta.

El sistema de coordenadas galácticas tiene al Sol como su origen y se compone de las coordenadas de longitud galáctica $l$ y latitud galáctica $b$. La longitud es un ángulo medido en sentido antihorario respecto al eje que se forma desde el Sol hasta el centro de la Galaxia, tomando valores en el rango $\ang{0}\le l\le\ang{360}$. La latitud es un ángulo en grados medido con respecto al plano galáctico y toma valores positivos al norte y negativos al sur, en un rango $\ang{-90}\le b\le\ang{90}$.

Los cuadrantes galácticos son la división de la Vía Láctea en cuatro sectores circulares, que se describen usando números ordinales o romanos y son: primer cuadrante galáctico, $\ang{0}\le l\le\ang{90}$; segundo cuadrante galáctico, $\ang{90}\le l\le\ang{180}$; tercer cuadrante galáctico, $\ang{180}\le l\le\ang{270}$; cuarto cuadrante galáctico, $\ang{270}\le l\le\ang{360}$.

El cuarto cuadrante galáctico es visible mayoritariamente solo desde el hemisferio sur y corresponde a la zona de observación para las mediciones provistas por el equipo docente.

Se tiene un cubo de datos FITS con mediciones de la transición rotacional fundamental en \SI{115}{\giga\hertz} del monóxido de carbono denominada CO(J=1$\to$0), para valores de 385 longitudes desde \ang{300} hasta \ang{348} equiespaciadas por \ang{0.125}, 33 latitudes de \ang{-2} a \ang{2} equiespaciadas por \ang{0.125} y 306 velocidades desde \SI{-230.7985}{\kilo\metre\per\second} hasta \SI{165.8235}{\kilo\metre\per\second} equiespaciadas por \SI{1.3004}{\kilo\metre\per\second}. La velocidad es una interpolación lineal que se puede obtener gracias al efecto Doppler, que establece, ${\Delta\nu}/{\nu_0}=-{\Delta v}/{c}$ , donde $\nu$ es la frecuencia observada, $\nu_0$ la frecuencia en reposo de la transición en laboratorio, $\Delta \nu$ su diferencia, $\Delta v$ la diferencia entre la velocidad radial de la fuente y la velocidad del receptor en la misma dirección, y $c$ la velocidad de la luz.

Los datos de temperatura se encuentran desplazados al azul, es decir, $\Delta v<0$, puesto que la fuente se está acercando en el cuarto cuadrante, ya que la galaxia está rotando y el gas entra y sale de los espirales con velocidad angular mayor en el centro que en los extremos, lo que se denomina rotación diferencial.

Las velocidades están medidas en el sistema de referencia LSR (\textit{Local Standard of Rest}, reposo local), para la velocidad de una partícula ficticia que se mueve alrededor del plano de la Vía Láctea en la órbita circular cerrada que pasa a través de la posición actual del Sol. Se usa la simplificación de movimiento puramente circula, aunque el Sol sigue una órbita de excentricidad menor a la décima.

La velocidad peculiar del Sol con respecto al LSR es aproximadamente \SI{16.5}{\kilo\metre\per\second} con movimiento propio hacia $l=\ang{55}$ y $b=\ang{25}$. Actualmente la IAU define la velocidad rotacional del Sol como $v_\odot^{\textnormal{rot}}=\SI{220}{\kilo\metre\per\second}$ con respecto al centro de la galaxia y en sentido horario si se observa desde el polo norte galáctico, y el radio galactocéntrico del Sol como $R_\odot=\SI{8.5}{\kilo\parsec}$.

La curva de rotación de la Galaxia permite estudiar su cinemática y características del material que está rotando, tal como la cantidad de vorticidad y \textit{shear} para cada radio galactocéntrico, que regulan la estabilidad gravitacional de la rotación diferencial del disco galáctico gaseoso, determinando así la forma y distribución del mismo a grandes escalas.

La derivación de la curva de rotación necesita determinar la velocidad terminal, que es la máxima velocidad absoluta de las velocidades radiales del lado del corrimiento al azul del espectro a cierta lontitud y latitud dadas y que cumple con ser cinco veces mayor al nivel de temperatura de ruido, permitiendo así para cada longitud obtener el máximo maximorum entre todas las latitudes. Esta velocidad es a lo largo de la línea de visión corresponden a puntos que, bajo la suposición de movimiento circular puro, son tangentes a las circunferencias galactocéntricas y se denominan puntos subcentrales.

Este informe primero deriva la curva de rotación galáctica, para luego obtener la corrugación de la Galaxia a partir de los puntos de la curva anterior y finalmente ajustar la curva a cinco modelos de distribución de masa de la Galaxia, obteniendo que el mejor conforma considera una masa puntual en el centro de la Galaxia y un disco uniforme alrededor. Se incluye un anexo con los códigos para los cálculos y gráficos, que también están disponibles en \url{https://github.com/camilo-nb/AS3201-tareas/tree/main/tarea2}.

\section{Hot--Cold Test}\label{sec:hotcoldtest}

\subsection{Marco teórico}

La señal de un objeto celeste recibida por un radiotelescopio se refleja en sus discos parabólicos para llegar al receptor espectrógrafo. El receptor tiene corrientes internas y fenómenos de transporte de electrones y fotones con ondas estacionarias que aumentan la entropía y generan un error sistemático correspondiente a ruido blanco.

El ruido blanco del receptor se puede disminuir al aumentar el tiempo de integración pues la señal del cielo permanece constante y la señal del receptor disminuye relativamente su tamaño, por lo que disminuir el ruido permite disminuir el tiempo invertido para detectar una señal astronómica, y a la vez permite no empeorar la señal. Además, se debe considerar que las horas de observación de un telescopio son muy costosas, por lo que es preferible minimizarlas.

Una manera común de eliminar errores sistemáticos es mediante la calibración del instrumento de medición.

El método Hot--Cold Test permite calibrar el telescopio al caracterizar la temperatura de ruido del receptor mediante dos cargas cuyas temperaturas son conocidas y diferentes.

Una carga es un material absorbente adherido a un pedazo de madera con un mango. El material es un absorbente electromagnético que absorbe la radiación con muy poca reflexión por lo que se supone como cuerpo negro. Si no fuera así, se formaría una onda estacionaria en la antena y las mediciones estarían fuertemente influidas por la posición de la carga.

Sea $T_\textnormal{rec}$ la temperatura de ruido del receptor y $G_\textnormal{rec}$ la ganancia del receptor. Una carga a temperatura ambiente $T_\textnormal{hot}$ se pone enfrente de la bocina de la antena, permitiendo medir una potencia espectral $W_\textnormal{hot}$ dada por,
\begin{equation}
W_\textnormal{hot}=G_\textnormal{rec}kT_\textnormal{rec}+G_\textnormal{rec}kT_\textnormal{hot}
,\end{equation}
donde $k$ es la constante de Boltzmann. Análogamente, para una carga fría a temperatura $T_\textnormal{cold}$, la potencia espectral medida es,
\begin{equation}
W_\textnormal{cold}=G_\textnormal{rec}kT_\textnormal{rec}+G_\textnormal{rec}kT_\textnormal{cold}
.\end{equation}
Se define el factor $Y$ como el cuociente entre la medición de la potencia espectral para la carga caliente y para la fría,
\begin{equation}
Y=\frac{W_\textnormal{hot}}{W_\textnormal{cold}}\label{eq:yfactor}
,\end{equation}
que permite determinar la temperatura de ruido del receptor mediante variables medidas,
\begin{equation}
T_\textnormal{rec}=\frac{T_\textnormal{hot}-YT_\textnormal{cold}}{Y-1}\label{eq:trec}
.\end{equation}
Se aprecia que un valor de $Y$ grande significa un $T_\textnormal{rec}$ pequeño, por lo que el receptor es más sensible, mientras que un valor cercano a la unidad significa que no es capaz de distinguir entre cargas calientes y frías, considerando que las cargas de referencia tienen temperaturas lo suficientemente disntantes.

En términos de operación del telescopio, la medición de $T_\textnormal{rec}$ se hace cada día que es observa, por lo que en una campaña de meses se tiene que hacer todos los días la medición. Esta constante medición corresponde a un chequeo del estado de la electrónica del telescopio pues si repentinamente difiere mucho la temperatura de ruido del receptor con respecto a la medición del día anterior es porque está funcionando mal y se debe arreglar.

Una propuesta astronómica debe establecer las horas necesarias de observación y para eso es necesario conocer la temperatura típica de ruido que determina el tiempo de la precisión.

En realidad, un radiotelescopio mide la potencia por canales de frecuencia y no la potencia espectral de todo el ancho de banda, tal como se describe en la sección \ref{sec:calibracion}, por lo que la temperatura de ruido del receptor cambia con la frecuencia.

\subsection{Datos y metodología}

Esta calibración utiliza una carga caliente a temperatura ambiente en la cúpula del radiotelescopio MINI y una carga fría a temperatura de nitrógeno líquido, cuyas mediciones están en la tabla \ref{tab:hotcoldtest}.

El MINI es pequeño, permitiendo acceder a la bocina por una escalera. Se sube y se pone la carga caliente enfrente de la bocina procurando apuntar el material absorbente a ella. Mediante un \textit{powermeter} se mide la potencia integrada de la señal para todo el ancho de banda que tiene el receptor y se anota la lectura en la tabla \ref{tab:hotcoldtest}.

A continuación y análogamente, se pone la carga fría enfrente de la bocina y se mide la potencia espectral con el \textit{powermeter} pero esperando a que la lectura correspondiente converja tras disminuir la temperatura. Esta medición está en la tabla \ref{tab:hotcoldtest}.

\begin{table}[p]
	\centering
	\begin{tabular}{
			@{}
			l
			S[table-format=3.0]
			S[table-format=2.2]
			@{}
		}
		\toprule
		{Carga} &
		{Temperatura} &
		{Potencia} \\
		{} &
		{\si{\kelvin}} &
		{\si{\dBm}} \\
		\midrule
		Hot & 300 & -44.50 \\
		Cold & 77 & -47.94 \\
		\bottomrule
	\end{tabular}
	\caption{Temperatura y potencia medidas para las cargas del Hot--Cold Test}\label{tab:hotcoldtest}
\end{table}

El \textit{powermeter} entrega potencias en decibelio-milivatio (\si{\dBm}), que es una escala logarítmica acorde a las eventuales amplificaciones y disminuciones de las señales. Una potencia $W$ en escala logarítmica de \si{\dBm} se convierte en una potencia $P$ en escala lineal de \si{\watt} como se muestra a continuación,
\begin{equation}
P=10^{\frac{W-3}{10}}\label{eq:dbm2w}
.\end{equation}

\subsection{Cálculo de $T_\textnormal{rec}$}

Se convierten las potencias de la tabla \ref{tab:hotcoldtest} a vatios según la ecuación \ref{eq:dbm2w} y se calcula $Y$ según la ecuación \ref{eq:yfactor}, obteniendo $Y=\num{2.2}$. Esto permite usar la ecuación \ref{eq:trec} para obtener $T_\textnormal{rec}=\SI{107.6}{\kelvin}$. Este cálculo se desarrolla en el código \ref{cod:hotcoldtest}.

\subsection{Comparación con calibración del MINI}\label{sec:calibracion}

El software del MINI tiene el comando \texttt{\%hct} para ingresar todo el sistema a una subrutina de Hot--Cold Test. Se usan cargas las mismas temperaturas de la tabla \ref{tab:hotcoldtest}.

El sistema espera a que se ponga la carga caliente en la bocina del receptor y se marca la medición al presionar una botonera, permitiendo tener una potencia por cada canal de frecuencia de la señal, mostrando una variación en todo el espectro. Ahora se pone la carga fría y se apreta la botonera, midiendo una potencia por cada canal que también varía en todo el espectro pero que es menor.

Se usan los dos vectores de potencia para calcular el factor Y según la ecuación \ref{eq:yfactor} y luego la temperatura de ruido del receptor por canal según la ecuación \ref{eq:trec}. Finalmente, se promedian las temperaturas de todos los canales, resultando una temperatura de ruido del receptor de $T_\textnormal{rec}'=\SI[separate-uncertainty=true]{150.9(46)}{\kelvin}$.

Se aprecia que $\abs{T_\textnormal{rec}'-T_\textnormal{rec}}=\SI[separate-uncertainty=true]{43.3(46)}{\kelvin}$.
\section{Antenna Dipping}\label{sec:antennadipping}

\subsection{Marco teórico}

La turbulencias atmosféricas y el vapor de agua presente en la atmósfera distorsionan la señal de un objeto celeste dando lugar a un error sistemático que no depende del radiotelescopio.

La atmósfera en sus distintas capas varía la temperatura y la densidad pero en este experimento se hace una aproximación a primer orden para establecer que contribuye una temperatura de ruido $T_\textnormal{atm}$ fija. Además, se hace también una aproximación al igualar la temperatura ambiente del domo $T_\textnormal{amb}$ con $T_\textnormal{atm}$.

La opacidad cenital $\tau_\textnormal{w}$ debido al contenido de agua en la atmósfera y la opacidad cenital total $\tau_0$ de la atmósfera, que incluye al oxígeno, se aproximan también a primer orden para decir que son iguales.

El método Antenna Dipping permite calibrar el receptor del telescopio al determinar el estado y opacidad cenital de la atmósfera.

Se mide la potencia del cielo $W_\textnormal{sky}$ detectada por el telescopio a un determinado intervalo de frecuencias y elevación $\varphi$, que está dada por,
\begin{equation}
W_\textnormal{sky}=c\left(T_\textnormal{rec}+T_\textnormal{atm}\left(1-\exp\left(-\tau_\textnormal{0}/\sin\varphi\right)\right)\right)
,\end{equation}
donde $T_\textnormal{rec}$ es la temperatura de ruido del receptor y $c$ es una constante que, por ejemplo, puede depender de la ganancia del receptor.

Análogamente, la potencia para una carga enfrente de la bocina a temperatura ambiente $T_\textnormal{amb}$ (la misma carga Hot del Hot--Cold Test), la potencia es,
\begin{equation}
W_\textnormal{amb}=c\left(T_\textnormal{rec}+T_\textnormal{amb}\right)
.\end{equation}

Aplicando las aproximaciones, se define,
\begin{equation}
\Delta W\equiv W_\textnormal{amb}-W_\textnormal{sky}=cT_\textnormal{amb}\exp(-\tau_\textnormal{w}/\sin\varphi)\label{eq:deltaw}
,\end{equation}
y $z=\pi/2-\varphi$, permitiendo obtener la relación,
\begin{equation}
\ln\left(\Delta W\right)=-\sec\left(z\right)\tau_\textnormal{w}+\ln\left(cT_\textnormal{amb}\right)\label{eq:taufit}
,\end{equation}
que es una relación lineal de $\ln(\Delta W)$ con respecto a ${-\sec\left(z\right)}$, donde $\tau_\textnormal{w}$ es la pendiente de la recta.

\subsection{Datos y metodología}

Se utiliza una carga caliente a temperatura ambiente en la cúpula del radiotelescopio MINI, midiendo la potencia que muestra Domo en la tabla \ref{tab:taufit}.

Se mide con un \textit{powermeter} la potencia espectral de diez puntos a distintas elevaciones y azimut fijo, entregando los resultados de la tabla \ref{tab:taufit}.
\begin{table}[htbp]
	\centering
	\begin{tabular}{
			@{}
			l
			S[table-format=2.2]
			S[table-format=2.2]
			@{}
		}
		\toprule
		{Punto} &
		{Elevación} &
		{Potencia} \\
		{} &
		{\textdegree} &
		{\si{\dBm}} \\
		\midrule
		1 & 23.50 & -45.56 \\
		2 & 16.60 & -45.22 \\
		3 & 12.84 & -45.01 \\
		4 & 10.48 & -44.88 \\
		5 & 8.85 & -44.80 \\
		6 & 7.66 & -44.73 \\
		7 & 6.76 & -44.71 \\
		8 & 6.04 & -44.67 \\
		9 & 5.47 & -44.65 \\
		10 & 4.99 & -44.63 \\
		Domo & & -44.54 \\
		\bottomrule
	\end{tabular}
	\caption{Elevación y potencia para los distintos puntos a azimut fijo. Se incluye domo con carga caliente}\label{tab:taufit}
\end{table}

\subsection{Cálculo de $\tau_\textnormal{w}$}

El siguiente procedimiento corresponde al código \ref{cod:antennadipping}. Se lee el archivo \texttt{antdip\_AE2021A} provisto por el equipo docente con los datos necesarios para realizar el ajuste lineal según la ecuación \ref{eq:taufit}. El gráfico de este se muestra en la figura \ref{fig:taufit}. La pendiente de la recta establece un valor de \num{0.2573501} para la opacidad cenital debido al contenido de agua en la atmósfera.

\begin{figure}[htbp]
	\centering
	\includegraphics{taufit.pdf}
	\caption{Equis roja es medición del telescopio. Línea azul es ajuste lineal. La pendiente es $\tau_\textnormal{w}$}
	\label{fig:taufit}
\end{figure}

\subsection{Comparación con calibración del MINI}

El software del MINI tiene el comando \texttt{\%antdip} para ingresar todo el sistema a una subrutina de Antenna Dipping.

Primeramente se debe ingresar la cantidad de masas de aire por punto y se usa \num{1}, que es el valor típico.

Luego el sistema recolecta automáticamente información para diez puntos separados por una masa de aire a través de la elevación a azimut fijo, midiendo la mayor elevación primero y luego disminuye progresivamente. El telescopio constantemente evalúa su posición y solo toma datos cuando está apuntando con cierta tolerancia a la coordenada determinada por el sistema de control.

La información recolectada corresponde a la diferencia de potencia de la ecuación \ref{eq:deltaw}, donde la potencia para el cielo es la que apunta según la elevación correspondiente y la potencia para la carga caliente es según el \textit{chopper}, una carga absorbente electromagnética, que está dentro de la bocina de la antena y automática y periódicamente obstruye la visión del telescopio gracias a un motor.

Tras medir los diez puntos, automáticamente el telescopio apunta al domo y toma una medición de referencia con la carga caliente. A continuación, se debe ingresar la temperatura actual y la humedad relativa, además de un parámetro cualitativo de \num{0} a \num{3} que indica el grado de cobertura del cielo debido a las nubes y sirve para el registro histórico.

Finalmente, el software realiza el ajuste lineal y entrega el valor $\tau_\textnormal{w}'=\num{0.2573501}$, además de otros parámetros como la estimación de la eficiencia del telescopio y la temperatura de brillo del agua en el cielo.

Se aprecia que $\tau_\textnormal{w}=\tau_\textnormal{w}'$.
\section{Observaciones}

\begin{figure}[htbp]
	\centering
	\includegraphics{lb.pdf}
	\caption{Coordenadas galácticas}
	\label{fig:lb}
\end{figure}

\begin{figure}[htbp]
	\centering
	\includegraphics{specfit1.pdf}
	\caption{Fit gaussiano}
	\label{fig:specfit1}
\end{figure}
\newpage

\section{Conclusiones}\label{sec:conclusiones}

En primer lugar, para el siguiente análisis se debe considerar que los datos provistos en archivos por el equipo docente corresponden a mediciones diferentes de las experiencias prácticas virtuales vistas por sus estudiantes, por lo que necesariamente cambian las condiciones del experimento y hay lugar a error. Se propone desde ya como mejora a este informe realizar una comparación con los datos correspondientes bajo la supervisión presencial en el radiotelescopio MINI.

Se concluye que la temperatura de ruido del receptor calculada para el radiotelescopio MINI, en las condiciones del experimento, cambia en un $31\%$ con respecto a la calibración propia del \textit{software}, lo que se puede considerar dentro de un rango aceptable para este parámetro.

La tabla \ref{tab:taufit} permite concluir que, a menor elevación, mayor es la potencia, pues la cantidad de atmósfera que se observa y, por lo tanto, su cantidad de emisión observada, es también mayor.

La opacidad calculada de la atmósfera difiere demasiado con respecto a la calibración propia del \textit{software} del telescopio, y se puede explicar porque un repentino cambio en el clima, tal como, por ejemplo, una nube, puede cambiar la opacidad de la atmósfera. Además, el modelo del \textit{software} tiene acceso a las mediciones por canal de ancho de banda y de por sí es más complejo, pues considera parámetros como humedad relativa, que son despreciados en la teoría del informe.

Las figuras \ref{fig:specfit1}, \ref{fig:specfit2} y \ref{fig:specfit3} muestran una alta relación señal a ruido, que se explica por la calibración hecha con los parámetros anteriores, junto con su correcta interpretación, pues permiten mejorar la precisión del radiotelescopio MINI al reducir el ruido, y por el tiempo de observación prolongado escogido. Estas figuras presentan una clara emisión en la nebulosa de Orión pues hay un gran pico en la temperatura, pero con la teoría provista en este curso no se puede saber \textit{a priori} los átomos emisores.

Las figuras \ref{fig:tmax} y \ref{fig:tint} muestran que el \textit{pointing} del MINI es aceptable para este experimento, pues la coordenada central de la cruz de observación es muy cercana al pico de temperatura de la fuente, que coincide con su centro y es la dirección hacia la que se ordena apuntar. Existe un pequeño error que se cree es debido a la tolerancia que permite el sistema de \textit{tracking} del radiotelescopio. Además, las temperaturas integradas permiten calcular la masa de la nube molecular pero los conocimiento provistos hasta ahora por este curso no permiten calcularla.

Finalmente, la tabla \ref{tab:rms} muestra que, por un lado, para la primera pasada por la cruz, el modelo teórico del error RMS coincide con las mediciones, pudiendo corroborar la relación $1/\sqrt{N}$ para este caso particular de tres datos promediados. Sin embargo, el método de observación sugiere que haya un factor adicional de $\sqrt{2}$ que no tiene lugar en los resultados. Por otro lado, la segunda pasada, que es temporalmente posterior a la primera pasada, tiene una mayor diferencia en el modelo teórico, y la tercera pasada resalta aún más esta diferencia, pero estos resultados tampoco se pueden explicar por la constante observacional de \textit{position switching}, por lo que se cree que un factor climático influyó fuertemente en las mediciones, como, por ejemplo, una nube moviéndose por la línea de visión.

\onecolumn

\section{Anexos}

\lstinputlisting[language=iPython, label={cod:vrot}, caption={Curva de rotación}]{rsc/curvaderotacion.py} 

\lstinputlisting[language=iPython, label={cod:Z}, caption={Corrugación del plano galáctico}]{rsc/corrugaciondelplano.py}

\lstinputlisting[language=iPython, label={cod:massmodel}, caption={Ajuste de los modelos de distribución de masa de la Galaxia}]{rsc/ajustedemodelodemasa.py} 

\end{document}