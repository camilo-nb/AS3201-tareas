\documentclass[letter, 10pt, twocolumn]{article}

\usepackage[utf8]{inputenc}
\usepackage[spanish]{babel}

\usepackage[left=0.75in,right=0.75in,top=0.75in,bottom=0.75in,columnsep=0.5in]{geometry}
\usepackage{graphicx}
\usepackage{float}
\usepackage{hyperref}

\usepackage{amsmath}
\numberwithin{equation}{section}

\usepackage{physics}

\usepackage{siunitx}
\sisetup{output-decimal-marker = {,}}
\DeclareSIUnit{\parsec}{pc}
\usepackage{booktabs}
\usepackage{scrextend}
\def\tablename{Tabla}
\usepackage{stfloats}

\input{rsc/listings.tex}
\usepackage{rsc/long2}
\usepackage{titling}

\title{Informe Instrumentación Astronómica}
\author{Camilo Núñez Barra}

\begin{document}
\longtwocolumn[{
\begin{figure}[H]
	\includegraphics[width=2in]{rsc/fcfm_das_eps}
\end{figure}
\begin{center}
	\textbf{\huge{INFORME\\CINEMÁTICA\\GALÁCTICA}}\\\,\\Tarea 2\\\,
\end{center}
\centering
\begin{tabular}{rl}
	Estudiante:&Camilo Núñez Barra\\
	RUT:&20.533.326-6\\
	Profesor:&Leonardo Bronfman\\
	Auxiliar:&Paulina Palma\\
	Curso:&AS3201 Astronomía Experimental\\
	Fecha:&\today\\
\end{tabular}
\vspace{1in}}]

\section{Introducción}

La visión que se tiene desde la Tierra de la Vía Láctea tiene franjas de oscuridad, que en un principio no se sabía si eran agujeros o materia interviniente. Esto se denomina el dilema de Bernard (1900) y fue resuelto en 1937 por Eddington con la conclusión de que es materia la que está absorbiendo.

La Galaxia está formada principalmente por hidrógeno molecular gaseoso $\textnormal{H}_2$, además de otros gases, polvo, muchas estrellas, sistemas solares, nebulosas y asteroides, todos agrupados gracias a la fuerza de gravedad existente. A menor escala, las nubes frías de hidrógeno gaseoso son importantes porque pueden indicar el lugar de nacimiento de estrellas debido a colapsos gravitacionales, además, en los brazos espirales se forman estrellas de alta masa. A gran escala, estas nubes delinean la estructura de la galaxia.

La molécula de hidrógeno no se puede observar directamente porque la atmósfera bloquea sus transiciones en el ultravioleta, además, es una molécula simétrica y su centro de masa coincide con su centro de carga, por lo que tiene momento dipolar nulo y la probabilidad de transiciones cuadripolares es muy baja.

El monóxido de carbono CO es la segunda molécula más abundante. En cambio, el CO tiene un momento de dipolo fuerte y transiciones  distintas frecuencias. La abundancia del CO relativa al H$_2$ es aproximadamente $10^{-4}$. La densidad de columna de hidrógeno $N(\textnormal{H}_2)$ es proporcional a la temperatura integrada de monóxido de carbono $W(\textnormal{CO})$ mediante el factor de conversión $\chi$ que se puede calcular y vale aproximadamente \SI{2e20}{\per\centi\metre\squared\per\kelvin\per\kilo\metre\second}, por lo tanto se puede usar el CO como trazador de H$_2$ pues mediante la medición de $W(CO)$ se obtiene la densidad de columna de hidrógeno y, junto con el área proyectada de la nube de gas, la cantidad de partículas de hidrógeno, que permite calcular directamente la masa de la nube.

El plano galáctico es donde se concentra la mayor parte de las estrellas de la Galaxia y además pasa por el centro de masa de esta.

El sistema de coordenadas galácticas tiene al Sol como su origen y se compone de las coordenadas de longitud galáctica $l$ y latitud galáctica $b$. La longitud es un ángulo medido en sentido antihorario respecto al eje que se forma desde el Sol hasta el centro de la Galaxia, tomando valores en el rango $\ang{0}\le l\le\ang{360}$. La latitud es un ángulo en grados medido con respecto al plano galáctico y toma valores positivos al norte y negativos al sur, en un rango $\ang{-90}\le b\le\ang{90}$.

Los cuadrantes galácticos son la división de la Vía Láctea en cuatro sectores circulares, que se describen usando números ordinales o romanos y son: primer cuadrante galáctico, $\ang{0}\le l\le\ang{90}$; segundo cuadrante galáctico, $\ang{90}\le l\le\ang{180}$; tercer cuadrante galáctico, $\ang{180}\le l\le\ang{270}$; cuarto cuadrante galáctico, $\ang{270}\le l\le\ang{360}$.

El cuarto cuadrante galáctico es visible mayoritariamente solo desde el hemisferio sur y corresponde a la zona de observación para las mediciones provistas por el equipo docente.

Se tiene un cubo de datos FITS con mediciones de la transición rotacional fundamental en \SI{115}{\giga\hertz} del monóxido de carbono denominada CO(J=1$\to$0), para valores de 385 longitudes desde \ang{300} hasta \ang{348} equiespaciadas por \ang{0.125}, 33 latitudes de \ang{-2} a \ang{2} equiespaciadas por \ang{0.125} y 306 velocidades desde \SI{-230.7985}{\kilo\metre\per\second} hasta \SI{165.8235}{\kilo\metre\per\second} equiespaciadas por \SI{1.3004}{\kilo\metre\per\second}. La velocidad es una interpolación lineal que se puede obtener gracias al efecto Doppler, que establece, ${\Delta\nu}/{\nu_0}=-{\Delta v}/{c}$ , donde $\nu$ es la frecuencia observada, $\nu_0$ la frecuencia en reposo de la transición en laboratorio, $\Delta \nu$ su diferencia, $\Delta v$ la diferencia entre la velocidad radial de la fuente y la velocidad del receptor en la misma dirección, y $c$ la velocidad de la luz.

Los datos de temperatura se encuentran desplazados al azul, es decir, $\Delta v<0$, puesto que la fuente se está acercando en el cuarto cuadrante, ya que la galaxia está rotando y el gas entra y sale de los espirales con velocidad angular mayor en el centro que en los extremos, lo que se denomina rotación diferencial.

Las velocidades están medidas en el sistema de referencia LSR (\textit{Local Standard of Rest}, reposo local), para la velocidad de una partícula ficticia que se mueve alrededor del plano de la Vía Láctea en la órbita circular cerrada que pasa a través de la posición actual del Sol. Se usa la simplificación de movimiento puramente circula, aunque el Sol sigue una órbita de excentricidad menor a la décima.

La velocidad peculiar del Sol con respecto al LSR es aproximadamente \SI{16.5}{\kilo\metre\per\second} con movimiento propio hacia $l=\ang{55}$ y $b=\ang{25}$. Actualmente la IAU define la velocidad rotacional del Sol como $v_\odot^{\textnormal{rot}}=\SI{220}{\kilo\metre\per\second}$ con respecto al centro de la galaxia y en sentido horario si se observa desde el polo norte galáctico, y el radio galactocéntrico del Sol como $R_\odot=\SI{8.5}{\kilo\parsec}$.

La curva de rotación de la Galaxia permite estudiar su cinemática y características del material que está rotando, tal como la cantidad de vorticidad y \textit{shear} para cada radio galactocéntrico, que regulan la estabilidad gravitacional de la rotación diferencial del disco galáctico gaseoso, determinando así la forma y distribución del mismo a grandes escalas.

La derivación de la curva de rotación necesita determinar la velocidad terminal, que es la máxima velocidad absoluta de las velocidades radiales del lado del corrimiento al azul del espectro a cierta lontitud y latitud dadas y que cumple con ser cinco veces mayor al nivel de temperatura de ruido, permitiendo así para cada longitud obtener el máximo maximorum entre todas las latitudes. Esta velocidad es a lo largo de la línea de visión corresponden a puntos que, bajo la suposición de movimiento circular puro, son tangentes a las circunferencias galactocéntricas y se denominan puntos subcentrales.

Este informe primero deriva la curva de rotación galáctica, para luego obtener la corrugación de la Galaxia a partir de los puntos de la curva anterior y finalmente ajustar la curva a cinco modelos de distribución de masa de la Galaxia, obteniendo que el mejor conforma considera una masa puntual en el centro de la Galaxia y un disco uniforme alrededor. Se incluye un anexo con los códigos para los cálculos y gráficos, que también están disponibles en \url{https://github.com/camilo-nb/AS3201-tareas/tree/main/tarea2}.

\section{Curva de rotación}

\subsection{Marco teórico}

Sea $P$ un punto a través de la línea de visión a longitud $l$ y de distancia galactocéntrica $R=R_\odot\sin l$. Se supone $P$ bajo un movimiento puramente circular con velocidad $\vec{v}(R)$, denominada velocidad rotacional, cuya componente a lo largo de la línea de visión, es decir, su velocidad radial medible a través del efecto Doppler, es $\vec{v}_\parallel$, formando un ángulo $\alpha$ entre ambas, por lo tanto, contrarrestando la velocidad del Sol $\vec{v}_\odot$ relativa a la línea de visión,
\begin{equation}
v_\textnormal{LSR}(R)=v(R)\cos\alpha-v_\odot\sin l
.\label{eq:vll}\end{equation}
Sea $\beta$ el ángulo interior que forma el Sol, el punto $P$ y el centro galáctico. Se aprecia que $\beta=90+\alpha$, y, usando el teorema del seno,
\begin{equation}
\frac{\sin l}{R}\equiv\frac{\sin\beta}{R_\odot}=\frac{\cos\alpha}{R_\odot}
,\end{equation}
de modo que la ecuación \ref{eq:vll} se reescribe como,
\begin{equation}
v_\textnormal{LSR}(R)=v(R)\frac{R_\odot\sin l}{R}-v_\odot\sin l
,\label{eq:nomaestra}
\end{equation}
y, definiendo la velocidad angular mediante la siguiente relación,
\begin{equation}
v(R)=\omega(R)R
,\end{equation}
se obtiene la ecuación maestra de la cinemática de la Galaxia,
\begin{equation}
v_\textnormal{LSR}(R)=\left(\omega(R)-\omega(R_\odot)\right)R_\odot\sin l
.\label{eq:maestra}\end{equation}

Para una misma velocidad $v_\textnormal{LSR}(R)$, existe una doble ambigüedad en la distancia heliocéntrica $D$ del punto $P$ que le corresponde, pues, dentro del círculo solar, la línea de visión intersecta a la circunferencia de radio $R$ en un punto cercano y en otro lejano. Esto no ocurre fuera del círculo solar.

Es importante determinar la distancia $D$ para poder calcular la densidad numérica y densidad de masa gracias a la luminosidad, y, si no se determina, solo se puede calcular la densidad de columna.

Sea $D_\textnormal{N}$ la distancia cercana y $D_\textnormal{F}$ la distancia lejana. Geométricamente, se obtiene,
\begin{equation}
D_{\substack{F\\N}}=R_\odot\cos l\pm\sqrt{R^2-R_\odot^2\sin^2l}
\end{equation}

La velocidad máxima $v_\textnormal{LSR}^{\max}(R)$ que se puede medir en la línea de visión ocurre en el punto $P$ sobre esta que tiene la menor distancia galactocéntrica $R_{\min}=R_\odot\sin l$, suponiendo $\omega$ monotónicamente decreciente con respecto a $R$. Este punto se denomina punto subcentral. Su distancia heliocéntrica es $D_\textnormal{N}=D_\textnormal{F}=R_\odot\cos l=D_{\tan}$ y su velocidad tangencial o rotacional es, gracias a la ecuación \ref{eq:nomaestra}.
\begin{equation}
v_\textnormal{rot}(R_\odot\sin l)=v_\textnormal{LSR}(R_\odot\sin l)+v_\odot\sin l
,\label{eq:vrot}\end{equation}
y su velocidad angular es,
\begin{equation}
\omega(R_\odot\sin l)=\frac{v_\textnormal{LSR}}{R_\odot\sin l}+\omega(R_\odot)
.\label{eq:wrot}\end{equation}

Las velocidades rotacionales permitidas en la galaxia interna, en donde $R<R_\odot$ y luego $\omega(R)>\omega(R_\odot)$, para el cuarto cuadrante galáctico, en donde las longitudes está en el rango $\ang{270}<l<\ang{360}$, de modo que $\sin l<0$, cumplen, según la ecuación \ref{eq:maestra}, con $v_{\textnormal{rot}}<v(R)<0$.

\subsection{Detalle del algoritmo}

La curva de rotación se deriva de las velocidades terminales del espectro de CO para cada longitud galáctica $l$. El cuarto cuadrante galáctico tiene emisión con corrimiento al azul, por lo tanto, se escogen las velocidades terminales más negativas posibles para cada espectro y que cumplan cierta condición que se explica a continuación. El procedimiento corresponde al código X.

Se utiliza la técnica de \textit{sigma clipping} para obtener la base de ruido de los espectros con un número de 5 desviaciones estándar para el límite de recorte tanto superior como inferior, además de las iteraciones necesarias para que el algoritmo converja. Suponiendo que el promedio del ruido se anula, entonces la temperatura de ruido o valor RMS del ruido es igual a la desviación estándar del mismo.

La velocidad terminal se define para cada longitud $l$ y para cada latitud $b$ como aquella velocidad del primer punto del espectro de emisión cuya temperatura sea cinco veces mayor a la temperatura de ruido, considerando primero las velocidades del lado del corrimiento al azul del espectro, es decir, recorriendo el espectro desde la velocidad más negativa a la más positiva. Esta condición se muestra en la figura \ref{fig:vterminal}.

A continuación, se escoge para cada longitud $l$ el máximo maximorum entre las velocidades terminales de todas las latitudes $b$.

\begin{figure}[p]
	\includegraphics{rsc/vterminal.pdf}
	\caption{Un típico espectro de emisión, para las coordenadas $l=\ang{314.385}$ y $b=\ang{-0.250}$, con la condición para elegir la velocidad terminal. La línea discontinua horizontal representa un nivel de $5\sigma$ de ruido. La primera temperatura del lado del corrimiento al azul en superar esta condición ocurre a una velocidad $v_\textnormal{LSR}=\SI{-52.644}{\kilo\metre\per\second}$, que corresponde a la línea discontinua vertical.}
	\label{fig:vterminal}
\end{figure}

\subsection{Curva de rotación: $v_{\textnormal{rot}}$ vs. $R$ y \mbox{$\omega$ vs. $R$}}

Se usa la ecuación \ref{eq:vrot} y los máximos maximorum de velocidades terminales para graficar en la figura \ref{fig:vrot} las velocidades rotacionales para cada longitud.

Se utiliza la ecuación \ref{eq:wrot} junto con los máximos maximorum de velocidades terminales para graficar en la figura \ref{fig:vrot} las velocidades angulares para cada longitud.

\begin{figure}[p]
	\includegraphics{rsc/vrot.pdf}
	\caption{Velocidad rotacional}
	\label{fig:vrot}
\end{figure}

\begin{figure}[p]
	\includegraphics{rsc/w.pdf}
	\caption{Curva de rotación. Velocidad angular}
	\label{fig:w}
\end{figure}
\section{Corrugación del plano}

\subsection{Marco teórico}

El disco galáctico no es plano, sino que tiene una corrugación de aproximadamente \SI{150}{\parsec}, que es mucho menor comparado con los aproximadamente \SI{15}{\kilo\parsec} de radio. La corrugación de la altura $Z$ a través la distancia galactocéntrica $R$ se puede asimilar a un parche de un tambor de espesor minimal, puesto que la Galaxia tiene una estructura ondulatoria que depende en coordenadas cilíndricas de $\theta$ y $r$ y que funciona como una onda espiral de densidad y en $R$ hay modos normales de vibración, luego, se puede expresar analíticamente como una composición de funciones de Bessel, pero para este informe se hace un análisis más sencillo a partir de los datos medidos.

Gracias al algoritmo de la sección anterior, se toma como la posición de densidad máxima para cada longitud $l$ a la latitud $b$ donde se encuentra el máximo maximorum de velocidad terminal.

Se sabe también de la sección anterior que en este punto subcentral tangente al círculo la distancia galactocéntrica es $D=R_\odot\sin l$ y la distancia heliocéntrica es $D=R_\odot\sin l$. Geométricamente, la altura para este punto respecto al ecuador galáctico es,
\begin{equation}
Z=R_\odot\sin l\tan b
,\label{eq:Z}\end{equation}
pudiendo aproximar $\tan b\approx b$ para un régimen de pequeñas oscilaciones.

\subsection{Detalle del algoritmo}

Se utiliza la latitud $b$ del máximo maximorum de velocidad para cada longitud $l$ del código \ref{cod:vrot} para calcular en el código \ref{cod:Z} la altura en cada longitud del disco según la ecuación \ref{eq:Z}. 

\subsection{Corrugación del plano: $Z$ vs. $R$}

La figura \ref{fig:Z} muestra la corrugación del disco galáctico en función del radio galactocéntrico.

\begin{figure}[htbp]
	\includegraphics{rsc/Z.pdf}
	\caption{Corrugación del plano. Altura $Z$ de cada punto subcentral con respecto a la distancia galactocéntrica (eje inferior) y la longitud correspondiente (eje superior).}
	\label{fig:Z}
\end{figure}

\section{Ajuste de modelo de masa}

\subsection{Marco teórico}

Se quiere encontrar una fórmula analítica para la curva de rotación basándose en un modelo físico que considere la distribución de masa de la galaxia.

Se iguala para cierta partícula de masa $m$ en rotación pura la fuerza de gravedad con la fuerza centrípeta,
\begin{equation}
G\frac{M(R)m}{R^2}=m\frac{v_\textnormal{rot}^2(R)}{R}
,\end{equation}
donde $G$ es la constante de gravitación universal y $M$ es la masa de la galaxia dentro de un radio $R$ determinado por la distancia galactocéntrica de la partícula. Luego, la velocidad de rotación de la curva de rotación de la figura \ref{eq:vrot} se puede modelar por,
\begin{equation}
v_\textnormal{rot}(R)=\sqrt{G\frac{M(R)}{R}}
.\label{eq:vmass}\end{equation}

Se estudian cinco modelos de distribución de masa para la galaxia, listados a continuación.
\begin{itemize}
\item Masa puntual en el centro de la galaxia. $M(R)=M_0$. $M_0$ es el parámetro libre de la masa puntual.

\item Disco uniforme. $M(R)=\pi R^2s_0$. $s_0$ es el parámetro libre de densidad superficial uniforme de masa.

\item Esfera uniforme. $M(R)=\frac{4}{3}\pi R^3\rho_0$. $\rho_0$ es el parámetro libre de densidad uniforme volumétrica de masa.

\item Disco uniforme con una masa puntual en el centro. $M(R)=\pi R^2s_0+M_0$. Dos parámetros libres, $s_0$ y $M_0$.

\item Esfera uniforme con una masa puntual en el centro. $M(R)=\frac{4}{3}\pi R^3\rho_0+M_0$. Dos parámetros libres, $rho_0$ y $M_0$.
\end{itemize}

\subsection{Detalla del algoritmo}

El código \ref{cod:massmodel} muestra el sigueinte procedimiento. Se evalúan los cinco modelos de distribución de masa en la ecuación \ref{eq:vmass} y se realiza un ajuste no lineal de mínimos cuadrados según el algoritmo de Levenberg-Marquardt para cada modelo, asumiendo una desviación estándar de la unidad para las velocidades de rotación. Además, se calcula el error RMS de cada modelo con respecto a las velocidades de rotación medidas para determinar el mejor ajuste.

\subsection{Ajuste de los modelos de la curva de rotación}

La tabla \ref{tab:massmodel} muestra el resultado del ajuste de mínimos cuadrados con los parámetros óptimos para cada modelo de distribución de masa de la Galaxia y el error RMS con respecto a la velocidad rotacional medida.

La figura \ref{fig:massmodel} muestra el gráfico de los ajustes de los cinco modelos de distribución de masa de la Galaxia.

\begin{table*}[htpb]
	\centering
	\makebox[\textwidth]{
	\begin{tabular}{lcccc}
		\toprule
		{Modelo} &
		{$M_0$} &
		{$s_0$} &
		{$\rho_0$} &
		{RMS} \\
		{} &
		{$\textnormal{M}_\odot$} &
		{$\textnormal{M}_\odot\,\textnormal{kpc}^{-2}$} &
		{$\textnormal{M}_\odot\,\textnormal{kpc}^{-3}$} &
		{\si{\kilo\meter\per\second}} \\
		\midrule
		Punto & $3.655\times10^{10}\pm1.321\times10^{9}$ & --- & --- & $68.738$ \\
		Disco & --- & $6.419\times10^{8}\pm5.501\times10^{6}$ & --- & $17.248$ \\
		Esfera & --- & --- & $8.569\times10^{7}\pm1.883\times10^6$ & $43.337$ \\
		Disco+Punto & $3.928\times10^{9}\pm3.422\times10^{8}$ & $5.814\times10^{8}\pm6.851\times10^{6}$ & --- & $14.749$ \\
		Esfera+Punto & $1.259\times10^{10}\pm4.343\times10^{8}$ & --- & $6.127\times10^{7}\pm1.112\times10^{6}$ & $21.326$ \\
		\bottomrule
	\end{tabular}}
	\caption{Parámetros y error obtenido del ajuste no lineal de míminos cuadrados para los modelos de distribución de masa de la Galaxia con respecto a la velocidad rotacional medida.}\label{tab:massmodel}
\end{table*}

\begin{figure}[htbp]
	\includegraphics{rsc/massmodels.pdf}
	\caption{Ajuste de los modelos de distribución de masa de la Galaxia con respecto a la curva de rotación medida. Línea negra es datos medidos. Línea naranja es modelo de masa puntual central. Línea verde es modelo de disco uniforme. Línea naranja es modelo de esfera uniforme. Línea azul es modelo de disco uniforme con masa puntual central. Línea morada es modelo de esfera uniforme con masa puntual central. Las cinco líneas de ajustes tienen un área respectiva que representa el error de los parámetros óptimos.}
	\label{fig:massmodel}
\end{figure}

\section{Análisis y conclusiones}

\onecolumn

\section{Anexos}

\lstinputlisting[language=iPython, label={cod:vrot}, caption={Curva de rotación}]{rsc/curvaderotacion.py} 

\lstinputlisting[language=iPython, label={cod:Z}, caption={Corrugación del plano galáctico}]{rsc/corrugaciondelplano.py}

\lstinputlisting[language=iPython, label={cod:massmodel}, caption={Ajuste de los modelos de distribución de masa de la Galaxia}]{rsc/ajustedemodelodemasa.py} 

\end{document}