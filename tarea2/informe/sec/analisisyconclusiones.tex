\section{Análisis y conclusiones}

La figura \ref{fig:vrot} muestra que la velocidad rotacional de los puntos subcentrales es globalmente creciente con respecto al radio galactocéntrico. Sin embargo, localmente se aprecian alternadas fluctuaciones de aproximadamente \SI{50}{\kilo\metre\per\second} hasta alcanzar un radio de \SI{4}{\kilo\parsec}, y esto se explica porque la línea de visión cercana centro de la Galaxia atraviesa más nubes de gas y en el centro se producen colapsos y expansiones, aumentando la distribución de velocidades.

A partir de aproximadamente \SI{4}{\kilo\parsec} de radio, la magnitud de las fluctuaciones de la velocidad rotacional disminuye, a la vez que se aumenta la velocidad.

La velocidad rotacional pareciera ser aproximadamente constante para longitudes menores a \ang{310}.

La figura \ref{fig:w} muestra que la curva de rotación concuerda con que la galaxia presenta una rotación diferencial, puesto que la velocidad angular de los puntos subcentrales disminuye con la distancia galactocéntrica. Se aprecia también el mismo fenómeno anterior de fluctuaciones para radios menores.

La mayor velocidad angular para radios menores indica que una rotación completa de esa zona de la galaxia demora 66370000 años, mientras que para radios mayores la rotación completa demora 199100000 años. Además, la tendencia de disminución se puede extrapolar a \SI{8.5}{\kilo\parsec} para concordar con la velocidad angular $v_\odot/R_\odot$ de \SI{0.838e-15}{\radian\per\second} del Sol, ya que los últimos puntos aumentan el radio de aproximadamente \SI{6}{\kilo\parsec} a \SI{7}{\kilo\parsec} disminuyendo la velocidad angular en \SI{0.2e-15}{\radian\per\second} a aproximadamente \SI{1e-15}{\radian\per\second}.

No se aprecia ningún punto muy desviado de la tendencia creciente de la figura \ref{fig:vrot} ni de la tendencia decreciente de la figura \ref{fig:w}, por lo que se afirma que el valor RMS del ruido es adecuado, evitando seleccionar un falso positivo para la velocidad terminal.

La figura \ref{fig:Z} muestra que la corrugación del disco galáctico tienen un máximo aproximado de \SI{150}{\parsec}, correspondiendo con la cifra sugerida por el profesor y equivaliendo a un $1\%$ del radio de la Galaxia, que así se puede considerar relativamente plana.

La corrugación presenta fuertes fluctuaciones distribuidas respecto al plano ecuatorial. Globalmente se aprecia una oscilación que se puede representar con una función sinusoidal que tenga un máximo para un radio de \SI{2}{\kilo\parsec}, luego un mínimo para \SI{3}{\kilo\parsec}, otro máximo para \SI{4.5}{\kilo\parsec}, otro mínimo para \SI{6}{\kilo\parsec} y finalmente un máximo para \SI{7}{\kilo\parsec}. Sin embargo, esta sinusoide idealiza demasiado la medición real que presenta fuertes y repentinas desviaciones.

La tabla \ref{tab:massmodel} muestra según el error RMS la siguiente clasificación en orden decreciente de mejor ajuste de modelo a medición: 1) Disco+Punto, 2) Disco, 3) Esfera+Punto, 4) Esfera, 5) Punto. Además, el error de los parámetros óptimos es de un orden de magnitud para la densidad volumétrica, dos órdenes de magnitud para la densidad superficial y un máximo de dos órdenes de magnitud para la masa puntual.

La distribución de masa de la Galaxia que mejor ajusta las mediciones de la velocidad de rotación de los puntos subcentrales es la que contempla una masa puntual de $(3.928\pm0.342)\times10^9\textnormal{M}_\odot$ en el centro y un disco de densidad superficial uniforme de $(5.814\pm0.068)\times10^8\textnormal{M}_\odot\,\textnormal{kpc}^{-2}$. El centro de la Galaxia presenta un objeto compacto supermasivo denominado Sagitario A$^\ast$ y que tiene una masa de $(4.154\pm0.014)\times10^6\textnormal{M}_\odot$ junto con un disco de acreción, para esta fuente, un agujero negro es la única explicación posible conocida, por lo que este modelo concuerda con la gran masa puntual del centro pero no se ajusta de buena manera al tamaño real.

La figura \ref{fig:massmodel} muestra que el modelo Punto es completamente contrario a la tendencia medida de la velocidad. El modelo de Esfera es lineal creciente pero no se ajusta bien pues se ve obligado a pasar por el origen de los ejes. El modelo de Esfera+Punto erróneamente es decreciente para radios menores. El modelo Disco pasa por los datos en un comienzo pero para los radios más grandes está sobre los datos reales. El modelo Disco+Punto se ve pasar por un promedio de los datos al principio, pero tampoco logra ajustar correctamente las últimas velocidades relativamente constantes, sin embargo, lo hace mejor que los otros.

Se propone como mejora al informe tomar mediciones para coordenadas galácticas de todo el cuarto cuadrante y posibles alrededores, puesto que el rango medido excluye longitudes fuera del círculo solar, aunque esto tal vez requiera de ajustar la teoría utilizada. Además, tener más datos del centro de la Galaxia para volver a poner a prueba a los modelos. Considerar órbitas elípticas en vez de circulares también se cree que pueda permitir explicar de mejor manera las mediciones.

Finalmente, gracias a las mediciones y discusiones concluye que:
\begin{enumerate}
\item[i.] la Galaxia tiene una rotación diferencial con velocidad angular decreciente con respecto al radio galactocéntrico;

\item[ii.] la Galaxia presenta una corrugación, a grandes rasgos oscilatoria, con amplitudes de aproximadamente $1\%$ de su radio;

\item[iii.] el modelo de distribución de masa que mejor ajusta la Galaxia es el que contempla una masa puntual en el centro y un disco de densidad uniforme alrededor de esta.
\end{enumerate}
