\section{Introducción}

La visión que se tiene desde la Tierra de la Vía Láctea tiene franjas de oscuridad, que en un principio no se sabía si eran agujeros o materia interviniente. Esto se denomina el dilema de Bernard (1900) y fue resuelto en 1937 por Eddington con la conclusión de que es materia la que está absorbiendo.

La Galaxia está formada principalmente por hidrógeno molecular gaseoso $\textnormal{H}_2$, además de otros gases, polvo, muchas estrellas, sistemas solares, nebulosas y asteroides, todos agrupados gracias a la fuerza de gravedad existente. A menor escala, las nubes frías de hidrógeno gaseoso son importantes porque pueden indicar el lugar de nacimiento de estrellas debido a colapsos gravitacionales, además, en los brazos espirales se forman estrellas de alta masa. A gran escala, estas nubes delinean la estructura de la galaxia.

La molécula de hidrógeno no se puede observar directamente porque la atmósfera bloquea sus transiciones en el ultravioleta, además, es una molécula simétrica y su centro de masa coincide con su centro de carga, por lo que tiene momento dipolar nulo y la probabilidad de transiciones cuadripolares es muy baja.

El monóxido de carbono CO es la segunda molécula más abundante. En cambio, el CO tiene un momento de dipolo fuerte y transiciones  distintas frecuencias. La abundancia del CO relativa al H$_2$ es aproximadamente $10^{-4}$. La densidad de columna de hidrógeno $N(\textnormal{H}_2)$ es proporcional a la temperatura integrada de monóxido de carbono $W(\textnormal{CO})$ mediante el factor de conversión $\chi$ que se puede calcular y vale aproximadamente \SI{2e20}{\per\centi\metre\squared\per\kelvin\per\kilo\metre\second}, por lo tanto se puede usar el CO como trazador de H$_2$ pues mediante la medición de $W(CO)$ se obtiene la densidad de columna de hidrógeno y, junto con el área proyectada de la nube de gas, la cantidad de partículas de hidrógeno, que permite calcular directamente la masa de la nube.

El plano galáctico es donde se concentra la mayor parte de las estrellas de la Galaxia y además pasa por el centro de masa de esta.

El sistema de coordenadas galácticas tiene al Sol como su origen y se compone de las coordenadas de longitud galáctica $l$ y latitud galáctica $b$. La longitud es un ángulo medido en sentido antihorario respecto al eje que se forma desde el Sol hasta el centro de la Galaxia, tomando valores en el rango $\ang{0}\le l\le\ang{360}$. La latitud es un ángulo en grados medido con respecto al plano galáctico y toma valores positivos al norte y negativos al sur, en un rango $\ang{-90}\le b\le\ang{90}$.

Los cuadrantes galácticos son la división de la Vía Láctea en cuatro sectores circulares, que se describen usando números ordinales o romanos y son: primer cuadrante galáctico, $\ang{0}\le l\le\ang{90}$; segundo cuadrante galáctico, $\ang{90}\le l\le\ang{180}$; tercer cuadrante galáctico, $\ang{180}\le l\le\ang{270}$; cuarto cuadrante galáctico, $\ang{270}\le l\le\ang{360}$.

El cuarto cuadrante galáctico es visible mayoritariamente solo desde el hemisferio sur y corresponde a la zona de observación para las mediciones provistas por el equipo docente.

Se tiene un cubo de datos FITS con mediciones de la transición rotacional fundamental en \SI{115}{\giga\hertz} del monóxido de carbono denominada CO(J=1$\to$0), para valores de 385 longitudes desde \ang{300} hasta \ang{348} equiespaciadas por \ang{0.125}, 33 latitudes de \ang{-2} a \ang{2} equiespaciadas por \ang{0.125} y 306 velocidades desde \SI{-230.7985}{\kilo\metre\per\second} hasta \SI{165.8235}{\kilo\metre\per\second} equiespaciadas por \SI{1.3004}{\kilo\metre\per\second}. La velocidad es una interpolación lineal que se puede obtener gracias al efecto Doppler, que establece, ${\Delta\nu}/{\nu_0}=-{\Delta v}/{c}$ , donde $\nu$ es la frecuencia observada, $\nu_0$ la frecuencia en reposo de la transición en laboratorio, $\Delta \nu$ su diferencia, $\Delta v$ la diferencia entre la velocidad radial de la fuente y la velocidad del receptor en la misma dirección, y $c$ la velocidad de la luz.

Los datos de temperatura se encuentran desplazados al azul, es decir, $\Delta v<0$, puesto que la fuente se está acercando en el cuarto cuadrante, ya que la galaxia está rotando y el gas entra y sale de los espirales con velocidad angular mayor en el centro que en los extremos, lo que se denomina rotación diferencial.

Las velocidades están medidas en el sistema de referencia LSR (\textit{Local Standard of Rest}, reposo local), para la velocidad de una partícula ficticia que se mueve alrededor del plano de la Vía Láctea en la órbita circular cerrada que pasa a través de la posición actual del Sol. Se usa la simplificación de movimiento puramente circula, aunque el Sol sigue una órbita de excentricidad menor a la décima.

La velocidad peculiar del Sol con respecto al LSR es aproximadamente \SI{16.5}{\kilo\metre\per\second} con movimiento propio hacia $l=\ang{55}$ y $b=\ang{25}$. Actualmente la IAU define la velocidad rotacional del Sol como $v_\odot^{\textnormal{rot}}=\SI{220}{\kilo\metre\per\second}$ con respecto al centro de la galaxia y en sentido horario si se observa desde el polo norte galáctico, y el radio galactocéntrico del Sol como $R_\odot=\SI{8.5}{\kilo\parsec}$.

La curva de rotación de la Galaxia permite estudiar su cinemática y características del material que está rotando, tal como la cantidad de vorticidad y \textit{shear} para cada radio galactocéntrico, que regulan la estabilidad gravitacional de la rotación diferencial del disco galáctico gaseoso, determinando así la forma y distribución del mismo a grandes escalas.

La derivación de la curva de rotación necesita determinar la velocidad terminal, que es la máxima velocidad absoluta de las velocidades radiales del lado del corrimiento al azul del espectro a cierta lontitud y latitud dadas y que cumple con ser cinco veces mayor al nivel de temperatura de ruido, permitiendo así para cada longitud obtener el máximo maximorum entre todas las latitudes. Esta velocidad es a lo largo de la línea de visión corresponden a puntos que, bajo la suposición de movimiento circular puro, son tangentes a las circunferencias galactocéntricas y se denominan puntos subcentrales.

Este informe primero deriva la curva de rotación galáctica, para luego obtener la corrugación de la Galaxia a partir de los puntos de la curva anterior y finalmente ajustar la curva a cinco modelos de distribución de masa de la Galaxia, obteniendo que el mejor conforma considera una masa puntual en el centro de la Galaxia y un disco uniforme alrededor. Se incluye un anexo con los códigos para los cálculos y gráficos, que también están disponibles en \url{https://github.com/camilo-nb/AS3201-tareas/tree/main/tarea2}.
